\section{Methoden zur deskriptiven statistischen Auswertung von Zeitreihen}

\subsubsection*{Sourcen:} 	org.unijena.regionWK.Statistik.DeskriptiveStatistik.java
\subsubsection*{Beschreibung:}
Das Paket DeskriptiveStatistik erm�glicht die Berechnung von Lageparametern, Streuungsparametern, Formparametern und Extremwerten.

Folgende Lageparameter k�nnen berechnet werden:
\begin{itemize}
 \item Arithmetisches Mittel,
 \item Median,
 \item erstes Quartil,
 \item drittes Quartil.
 \end{itemize}
 
Zur Beschreibung der Streuung der Zeitreihe stehen folgende Ma�e zur Verf�gung:
\begin{itemize}
 \item Spannweite,
 \item Varianz,
 \item Standardabweichung,
 \item durchschnittliche Abweichung.
\end{itemize}

Formparameter beschreiben die Form der Verteilungsfunktion einer Zeitreihe und deren Abweichung von der Normalverteilung.

�ber die Schiefe ($g$) wird beschrieben ob und wie stark eine Verteilung in ihrer Symmetrie von der Normalverteilung abweicht. 

\begin{tabular}{ll}
Es gilt: &  \\ 
$g = 0$  & f�r symmetrische Verteilungen (Median = Mittelwert),\\
$g > 0$  & f�r linkssteile (= rechtsschiefe) Verteilungen ($\mathrm{Median < Mittelwert}$), \\
$g < 0$  & f�r rechtssteile (= linksschiefe) Verteilungen ($\mathrm{Median > Mittelwert}$). \\
\end{tabular}\\ 

Der Exzess ($Ez$) beschreibt ob und wie stark eine Verteilung hinsichtlich ihrer W�lbung von der Normalverteilung (Exzess = 3) abweicht.

\begin{tabular}{ll}
Es gilt: &  \\ 
$Ez = 0$  & gleiche W�lbung wie Normalverteilung,\\
$Ez > 0$  & geringere W�lbung als Normalverteilung, \\
$Ez < 0$  & st�rkere W�lbung als Normalverteilung. \\
\end{tabular}\\ 

Die Extrema einer Zeitreihe werden durch die Berechnung von Minimum und Maximum abgebildet.


